\chapter{Motivation}

\section{The need for change}
I politely ask you to consider: a typical child grows up being told over and over again to ``be quiet'' and to learn the materials presented in front of them; there are no stupid questions as long as they are relevant to the material at hand. As they progress into higher education (if they do), they find themselves often sitting in a large lecture hall, questions left unanswered ``in the interest of time,'' and wondering whether the lecturer knows their name, or even cares if they show up at all.

A Google search for ``changing higher education'' returns a plethora of articles responding to recent student strikes advocating for change in policies regarding finances and other economic concerns. In contrast, a search for ``changing K-12 education'' returns a 20-page ERIC (Education Resources Information Center) document on curriculum reform and effect on entering post-secondary institutions. Despite advances in research in education, classroom instruction has not changed significantly in a typical college in the last five to ten decades, if not more. Students are still attending large lecture-style classes assessed via exams accompanied by weekly, graded assignments. With an increasing number of individuals pursuing a higher education, it seems naive to think that the current systems of education are still suitable or equitable to all students. With the diversifying population of students, it is just as crucial to bring sensible change to promote an equitable learning setting.

\section{Mathematics is not fair}
The current nature of mathematics education in the US provides an extremely different experience for a student who is Asian American versus African American versus white. Starting from achievement gaps between African American and Latino/a students and white students to  the extremely stereotypical belief that ``Asians are good at math,'' there is abundant evidence for the unfair nature of mathematics education (\cite{alfinio_flores_examining_2007}).

Specifically, there exists a large socioeconomic difference between ethnicities. The 2007-2011 census provides enough quantitative evidence of the unequal distribution of poverty among different races. American Indians and African Americans came in the highest at about 26\% of the population being in poverty, more than a double in comparison to the 11.6\% of whites (\cite{macartney}). A 2018 New York Times article showcasing a megastudy conducted on white and black men showed that of the 5,000 white and 5,000 black boys who grew up in poverty, 48\% of black boys grew up to remain in poverty and only 2\% grew to be rich, while 31\% of white boys remaining in poverty and 10\% became rich (\cite{badger_extensive_2018}).

With income gaps this big, opportunity gaps are not so different. In 2013, enrollment percentages in postsecondary education showed about a near 10\% difference between white (42\%) and black (34\%) students (\cite{musu-gillette_status_nodate}). Graduation rates were similar, lowest for black students at around 41\%. These numbers dip down further for STEM (Science, Technology, Engineering, Mathematics) degrees, about 11\% apart.

Exactly what perpetuates these depressing statistics rests deeply rooted in racism that has lined all of US history. To be more specific, it is absolute ignorance to think that any part of education is void of racism, whether outwardly intentional or not. From the 2005 study conducted by the American Mathematics Society, 80\% of full time mathematics professors with PhDs are white, compared to 1\% black and 2\% Hispanic.

Yet, mathematics is historically not a unique subject to white Europeans; rather, prominent advancements were made by individuals from all over world. Yet, if we ask ourselves the names of famous mathematicians, what we hear are not Srinivasa Ramanujan, Hypatia, or Dorothy Vaughn but rather Euler, Pythagoras and Fermat. The problem in question lies exactly here--whiteness is rarely questioned in this context of mathematics. Without having stood in the shadow of an individual that society pictures to be the ``model mathematician,'' it is extremely difficult to understand the place of inequality in the education of our students. It is the not the white teacher but the student of color who has the sole responsibility to succeed through societal oppression. It is no different than asking an ugly duckling to be something other than what he sees in the reflection.

This lack of having a proper role model impacts the belief a student has that they can succeed, otherwise known as self-efficacy (\cite{thevenin_mentors_2007}). With lowered self-efficacy comes lowered achievement, unsurprisingly (\cite{motlagh_relationship_2011}). Once again, we see how the question of how these unfair societal norms factor into reducing the quality of education or effectiveness of education a student receives is rarely raised.

The solution is not clear either. There are many variables and factors in any classroom environment that either cannot be controlled or are unknown. There is no magic wand that turns poverty-stricken neighborhoods into more privileged ones, nor is there one that removes racial inequality within a school, never-mind school districts across a nation. One important example of an unexpected factor that aggravates the situation is microaggression, which describes any seemingly small behavior, including unvocalized assumptions, that relays hostility or prejudiced views towards a marginalized group, unintentional or not. When unnoticed or ignored, microaggressions towards ethnic minority groups feed racism, fueling a mindset that only continues to be confirmed as a correct one. As a result, impacted students fall further into the mindset of feeling less capable in the classroom.

In particular, this happens in subjects like mathematics, widely believed to be a neutral subject. First, neutrality on its own deserves to be revisited. Statements such as ``all lives matter'' made as a counter to the Black Lives Matter movement, describe a harmful form of using neutrality to deny racism exists. In a similar vein, stating that ``mathematics has no color'' contains as many subliminal messages. As Rochelle Gutierrez, notable for her advocacy in equitable education, writes:
\begin{displayquote}
  {[In]} many mathematics classrooms, students are expected to leave their emotions, their bodies, their cultures, and their values outside the classroom walls, stripping them of a sense of wholeness (\cite{gutierrez_embracing_2012}).
\end{displayquote}
It is not to say that finding a derivative is promoting white supremacy. Rather, the way in which we teach a mathematical concept and the myriad of assumptions we make in the process shape the role mathematics takes in our society. In this light, remodeling education to eliminate inequitable practices in mathematics seems a daunting task for any one nation, let alone an institution, to tackle.

Thus, I propose to look into establishing self-efficacy in students. The status quo continues to marginalize students, undermining their abilities and discouraging them from the simple thought that they can succeed in mathematics. By letting students believe that they are capable and entrusting them with their own capabilities, classrooms will no longer be a place for oppression but a place for opportunity for everyone. Building self-efficacy will hence build supportive environments that can empower students with independence and trust in themselves.

While I believe such an ambition can be achieved in various ways, not excluding political and administrative actions, I think fostering self-regulation is a promising first step. in the next chapter, self-regulation is defined and detailed, showing how it can build an equitable classroom.
