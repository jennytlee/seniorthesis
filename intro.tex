\chapter{Introduction}
\section{Motivation}

Before I begin, I ask the reader to consider my abbreviated attempt in describing my experience of mathematics education: a typical child sits in a classroom full of students, all who are told to practice and memorize the materials presented by the teacher; there are no stupid questions as long as they are relevant to the material at hand. As she attends her first classes in college, she again finds herself sitting in a classroom full of students, questions left unanswered ``in the interest of time.'' She wonders whether the lecturer knows her name, or even cares if she shows up at all.

Mathematics is a difficult subject for me. In large part, it is difficult because of how difficult it is to stay awake during lectures. Laughable, but it's true; I often joke that had it not been for my persistent parents' efforts and lifelong interest in the subject, I would have easily been an art major, which requires a lot more moving of body parts and far fewer listening of lectures. I am therefore not surprised when many of my peers express distress or dislike of math of any kind, or when in turn they express surprise to my liking of it. It is, in my experience, a difficult subject to learn to enjoy in the classroom.

I am therefore as baffled as I am frustrated with the institutional traditions that have existed in obstruction to effective learning. Forgetting for a moment that lectures are simply difficult to make interesting for over an hour, I, a female student of color, rarely see a figure for me to look up to or relate to. I have sat in numerous math lectures being the sole female student in the room, questioning where all of my fellow female math majors could have gone. I have shuffled through mathematics texts wondering when the last time I read a textbook from a female author of color was, never mind see a theorem named after one.

This thesis is my attempt to bring some of the problems I see into light, as a way to expose the flaws and changes necessary in mathematics pedagogy, particularly in postsecondary education. Afterwards, I propose a question that takes into consideration a way that may solve a subset of these problems using a small case study at my own institution.

\section{Background}
\subsection{The status quo.}

The classroom setting that I described in the very beginning is an example of the ``banking model'' of education, named by Paulo Freire in {\it Pedagogy of the Oppressed}, written in 1968. Freire draws a metaphor between this style of education to a bank, where the teacher is the depositor and the students are depositories (Friere, 1968). The teacher's tasks are to ``fill'' the students of information, and the student's role is to accept this information without questioning the contents. Thus, under this model, a good teacher is one who can give as much as they can to as many students, and a good student is one who receives and regurgitates the most with precision. Of the various problems Freire points out about this model, he particularly hammers down the notion that the banking model transforms students into objects that merely act as containers, devoid of critical or creative thought.

Friere is certainly not the only individual questioning the method of education utilized in classrooms today. Educator and psychologist Donald Bligh presents in his book, {\it What's the Use of Lectures?}, rationale for why traditional lecture style classrooms are ineffective and outdated, supported by an exhaustive collection of studies (Bligh, 1998). His ideas are not new either, compiling hundreds of studies conducted on this topic and relating theories from many other educators. Most notably, he cites the work of Benjamin Bloom, also a professor known for proposing and driving experiments on mastery-based learning (Bligh, 1998).

Despite the existing literature about the flaws of the traditional system, there have been resistance against change, in large part due to the impracticality and difficulty of bringing changes into fruition. Bob Moses describes his experience in spreading his nontraditional and new way of teaching algebra, called the Algebra Project, across middle schools in Boston as an ``uphill slough'' (Moses, Cobb, 2001). He was turned down by principals for reasons ranging from claiming student skills not being up to the level needed to stating that it would be too difficult to transition from the traditional approach.

Yet postsecondary mathematics in America looks not so different from a century ago. In fact, I claim this lecture-based, transmissive style of education still dominates mathematics classrooms today. A Google search for ``changing higher education'' returns a plethora of articles responding to recent student strikes advocating for change in policies regarding finances and other economic concerns. In contrast, a search for ``changing K-12 education'' returns a 20-page ERIC (Education Resources Information Center) document on curriculum reform and effect on entering post-secondary institutions. It was only in 2014 when an official committee by the name of Transforming Post Secondary Education Math was formed to in part discuss using a variety of instructive methods in the classroom (Dewar, Hsu, Pollack, 2016).

Nevertheless, the existence of this committee goes to show that there is persistent push for change and experimentation of different methods of instruction.
\subsection{Pedagogical methods}
Over the course of the century, a variety of pedagogical methods came into trial across subject areas and institutions. Specifically regarding postsecondary mathematics education, the following three methods have been examined.

\subsubsection{Flipped classrooms}
This method of instruction involving an inversion or a ``flip'' in the classroom is best described as placing the students in a setting where the lectures are given outside of the classroom, and activities meant to be more meaningful for the learning experience take place inside the classroom (Zappe, et al., 2009). These activities, ranging from group work to solving tutorials to leading workshops, are often described as active learning and have been shown to significantly improve student performance in science and mathematics (Freeman, et al., 2014).

As with all of the different methods tested, flipped classrooms are best explained by example. In a four-year experiment conducted by three professors at a liberal arts college, a flipped classroom model was used in an undergraduate differential equations class (Yong, Levy, Lape, 2015). Students in the flipped classrooms were required to watch a series of video lectures which were narrated with the instructors' voices as a part of their assigned homework. Then, during class, students worked on problems from the homework assignments, in particular ones that would involve mathematical modeling. They were encouraged to work in small groups, and instructors provided guidance and help so as to avoid common misconceptions or mistakes that could occur in tackling these problems.


\subsubsection{Inquiry-based learning}

\subsubsection{Self-regulation}


\subsection{Inequalities}
\section{Self-paced assessment}
