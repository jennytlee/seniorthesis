\chapter{Introduction}
\section{Motivation}

Before I begin, I ask you, reader, to consider my abbreviated description of my experience of mathematics education: a typical child sits in a classroom full of students, all who are told to practice and memorize the materials presented by the teacher; there are no stupid questions as long as they are relevant to the material at hand. As she attends her first classes in college, she again finds herself sitting in a classroom full of students, questions left unanswered ``in the interest of time.'' She wonders whether the lecturer knows her name or even cares if she shows up at all.

Mathematics has always been a difficult subject for me. In large part, it is difficult because of how difficult it is to stay awake during lectures. Perhaps laughable, but it's true; I often joke that had it not been for my persistent parents' efforts and lifelong interest in the subject, I would have easily been an art major, which requires a lot more moving of body parts and far fewer listening to lectures. I am therefore not surprised when many of my peers express distress or dislike of math of any kind, or when in turn they express surprise to my liking of it. It is, in my experience, a difficult subject to learn to enjoy in the classroom.

I am therefore as baffled as I am frustrated with institutional traditions that exist as obstructions to effective learning, both of mine and my peers'. Forgetting for a moment that lectures are simply difficult to be interesting for over an hour, I, a female student of color, rarely see a figure for me to look up to or relate to. I have sat in numerous math lectures being the sole female student in the room, questioning where all of my fellow female math majors could have gone. I have shuffled through mathematics texts wondering when the last time I read a textbook from a female author of color was, never mind see a theorem named after one. Being Asian-American, I constantly question and face biases and stereotypes of Asian-Americans, many which make me take a second glance my pride and love for my ethnical identity.

This list can go on, and I'm not happy. I find that this system of learning which disregards the students in the picture is highly ineffective; the context in which mathematics lives in for people like myself is simply non-negligible in fostering a good learning environment.

This thesis is my attempt to bring some of the problems I see into light, as a way to expose the flaws and changes necessary in mathematics pedagogy, particularly in postsecondary education. Afterwards, I propose a question that takes into consideration a way that may solve a subset of these problems using a small case study at my own institution.

\section{Background}
\subsection{The status quo.}

The classroom setting that I described in the very beginning is an example of the ``banking model'' of education, named by Paulo Freire in {\it Pedagogy of the Oppressed} in 1968. Freire draws a metaphor between this style of education to a bank, where the teacher is the depositor and the students are depositories (Friere, 1968). The teacher's tasks are to ``fill'' the students of information, and the student's role is to simply accept this information, with no particular requirement to digest its contents further or apply additional context. Thus, under this model, a good teacher is one who can give as much as they can to as many students, and a good student is one who receives and regurgitates the most with precision.

Of the various problems Freire points out about this model, he particularly hammers down the notion that the banking model transforms students into objects that merely act as containers, devoid of critical or creative thought.

Friere is certainly not the only individual questioning the method of education utilized in classrooms today. Educator and psychologist Donald Bligh presents in his book, {\it What's the Use of Lectures?}, rationale for why traditional lecture style classrooms are ineffective and outdated, supported by an exhaustive collection of studies (Bligh, 1998). His ideas are not new either, compiling hundreds of studies conducted on this topic and relating theories from many other educators. Most notably, he cites the work of Benjamin Bloom, also a professor known for proposing and driving experiments on mastery-based learning (Bligh, 1998).

Despite the existing literature about the flaws of the traditional system, there have been resistance against change, in large part due to the impracticality and difficulty of bringing changes into fruition. Take for instance educator Bob Moses, who saw struggling students in mathematics and created a nontraditional way of teaching algebra, called the Algebra Project. In his book, he describes the experience in spreading this idea across middle schools in Boston as an ``uphill slough'' (Moses, Cobb, 2001). He was turned down by principals for reasons ranging from teachers claiming that student skills not being up to par to stating that it would be too difficult to transition from the traditional approach.

Moses' experience goes to show that there exists resistance towards change, and isn't uncommon in middle school classrooms. Thus it is unlikely that postsecondary mathematics in America has changed since a century ago. In fact, I claim this lecture-based, transmissive style of education still dominates mathematics classrooms today. A Google search for ``changing higher education'' returns a plethora of articles responding to recent student strikes advocating for change in policies regarding finances and other economic concerns. In contrast, a search for ``changing K-12 education'' returns a 20-page ERIC (Education Resources Information Center) document on curriculum reform and effect on entering post-secondary institutions. It was only in 2014 when an official committee by the name of Transforming Post Secondary Education Math was formed to in part discuss using a variety of instructive methods in the classroom (Holm, 2016).

In some ways, the formation and existence of this committee reflects the  persistent push for change and experimentation of different methods of instruction.
\subsection{Pedagogical methods}
Throughout the 20th century, a variety of pedagogical methods came into trial across subject areas and institutions. Specifically in regards to postsecondary mathematics education, methods that have been explored include the following.

\subsubsection{Flipped classrooms}
This method of instruction involving an inversion or a ``flip'' in the classroom is best described as placing the students in a setting where the lectures are given outside of the classroom, and activities meant to be more meaningful for the learning experience take place inside the classroom (Zappe, et al., 2009). These activities, ranging from group work to solving tutorials to leading workshops, are often described as active learning and have been shown to significantly improve student performance in science and mathematics (Freeman, et al., 2014). There are ongoing studies on the effects of flipped classroom being done in undergraduate engineering and mathematics settings (Yong, Levy, Lape, 2015).

\subsubsection{Inquiry-based learning}
The Moore Method of teaching, which departs entirely from lectures and turns to having student-led classes that required the work in proving theorems and concepts be done by the students, was both hailed for its innovative ways to foster learning and criticized for unaccounted fallbacks and impracticalities with broader audiences (Parker, 1992).

Seeing ways he could improve the method, David Cohen introduced and experimented with the Modified Moore Method within mathematics classrooms, where he looked to reduce the downsides of the Moore Method while maintaining its intentions and benefits (Cohen 1982). Inquiry-based learning (IBL) stems here, particularly relating how students can learn actively through questioning and creating arguments independently (Yoshinobu, 2013).

IBL in postsecondary mathematics have recently received attention as they appeared in discussions and examples presented in an issue of {\it Problems, Resources, and Issues in Mathematics undergraduate Studies} (Katz, 2017). Various examples that appear in this issue of IBL studies, which were run on various student bodies and subjects, bring up notable questions and results.

\subsubsection{A focus on self-regulation}
Educator and psychologist Benjamin Bloom developed mastery-based learning, which expects students to have complete or near-complete mastery of concepts before proceeding further into the material (Bloom, 1968). Over the years mastery learning has taken on many variations, some more successful than others, but all focus on individual pacing and developing autonomy in a student's ability to learn. The biggest takeaway from studies done in mastery learning is the positive impact it has on students despite how much it differs from traditional methods of teaching (\cite{zollinger_impact_2017}, \cite{bradley_evaluating_2017}).

Self-regulation is not a method but a vital piece of mastery-based learning that asks students to identify and understand their own progression in learning. As described by Zeider, Pintrich and Boekaerts in the \emph{Handbook of Self-Regulation}:
\begin{displayquote}
  Self-regulation refers to self-generated thoughts, feelings, and actions that are planned and cyclically adapted to the attainment of personal goals (\cite{zeider}).
\end{displayquote}


\subsection{Inequalities}
\section{Self-paced assessment}
