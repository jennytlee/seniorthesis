\chapter{The case study}
\section{Introduction}

Before getting into the details of the case study, let me introduce you to my experience of the US College Education at Harvey Mudd College: projector screens, chalkboards, individual desks and syllabi stating exact dates to assignments and exams. Variety in student experience in liberal arts colleges in the US is not pronounced. In many of these colleges, large lecture halls are ready to be filled with hundreds of students for them to watch a single professor or instructor. Whether a thousand-person introductory course or a ten-person advanced class, a student is expected to consume the material and spit it out, correctly. This is not to ridicule the efforts of certain colleges that are trying to actively reform education, but still the vast majority has remained stagnant.

As mentioned in the earlier chapter, the focus of the study was to see if positive change could be brought about in the status quo via self-regulation--specifically, through the use of self-paced assessments. The case study was designed as an attempt to study direct effects of making a small change and adding an element of self regulation in a mathematics course. To do so, self-regulated, self-paced assessments were put in place of midterm and final examinations.

\section{Method}
This study focused on a mandatory introductory linear algebra course, listed as Math 40, offered as a part of the common graduation requirements. Math 40 is a 7 week course, and historically the course consists of about 10 homework assignments and two exams, including one midterm and one final. There were total 49 students involved being first year students, and students were split between 2 sections randomly, taught by the same instructor. One of the sections used self-paced assessments. The other section remained unchanged as a control. For ease in distinguishing the two, I will refer to the test section as the ``quiz section'' and the unchanged section as the ``control section.''

The 24 students of the quiz section did not have any midterm or final examinations. Instead, these students were required to finish a total of 10 small, one or two problem quizzes by the end of the course. All 10 closed-book, closed-notes assessments consisted of questions that pertained to the knowledge of the material that was taught up to the day of release. Students were able to retake these quizzes without penalty, but the ultimate grade of the quiz would be determined by the attempt with the highest score. There were no deadlines to any of these quizzes except for the final deadline at the very end of the semester. Students were also expected to finish each within 15 minutes. In other words, all quizzes were self-paced and take-home, meaning students had autonomy over when, where, and how they wished to take these assessments.

The control section took one midterm and one final exam as traditionally done, appropriately scheduled around half way and at the end of the semester. Homework assignments remained identical to the other sections, and instruction was essentially identical for the two sections under the same instructor. There were no extra or additional assignments, nor were there fewer assignments, in any one of the sections.

There were six other sections of Math 40 being taught at the same time by other instructors, but other than the same material being taught, had nothing to do with the study. Students were not given the choice to opt out of the study while in either the quiz or control section, but could choose to drop or switch into a different section not part of the study. No students in other sections were allowed to switch into either sections in study.

Students in both sections were asked to fill out a pre-survey during the first week of the course asking for information including demographics, high school math courses, and family backgrounds. This survey also asked for self-confidence and assessment in mathematics ability, belief in the need for certain elements in creating an intellectual environment, and level of comfort in asking questions.

After the course ended, students were asked to fill out a post-survey, which included questions about students' perceived growth in mathematics ability and confidence. They were also asked to participate in a Focus Group session for qualitative feedback using questions presented by a non-interactive individual not part of the study (neither I nor the instructor were present in the room).

\section{Results}

During the pre-test, students were asked for information on their demographics. Below, table \ref{table:case_ethnicity} gives an overview of the students' racial/ethnic identities.

\begin{table}[!htb]
  \begin{center}
    \begin{tabular}{l | l | l | l | l | p{1.2cm}}
      Section & Asian & Black & Hispanic & White & AIAN/ NHPI*\\ \hline
      Control & 6 & 0 & 5 & 18 & 3\\
      Quiz & 7 & 5 & 8 & 16 & 0\\
    \end{tabular}
  \end{center}
  {\footnotesize *American Indians and Alaskan Natives / Native Hawaiians and other Pacific Islanders}
  \caption{Number of students per each racial/ethnic group. Total is greater than sample size (accounting students of mixed race).}
  \label{table:case_ethnicity}
\end{table}

Overall, the pre- and post-surveys found little statistically significant evidence of differences between the two sections. Qualitatively, there were notable differences in descriptions of the experience in the course; in particular, many students in the quiz section noted a lower level of perceived stress.

Using $\alpha = 0.05$, two-tailed $t$-tests were run on the differences in averages were taken regarding student scores of self-assessment of math knowledge and self-perceived growth in confidence and knowledge of mathematics.

\subsubsection{1. $H_0$: Students self-assessed scores their math did not change (before $=$ after).}

Students were asked to assess their own knowledge of mathematics in both the pre- and post-surveys. Table \ref{table:means} shows the difference in averages of how students assessed their knowledge before and after the class.

\begin{table}[!htb]
  \begin{center}
    \begin{tabular}{l | c | c | c | c}
      Section & Before & After & Difference & $p$-value\\
      \hline
      Control & 3.164 & 3.160 & 0.024 & 0.890\\
      Quiz & 3.136 & 3.208 & 0.0720 & 0.652
    \end{tabular}
  \end{center}
  \caption{Differences in average of how students rated themselves on self assessment of math knowledge (lower value indicating lower score).}
  \label{table:means}
\end{table}

In both sections, the hypothesis $H_0$ was not rejected.

\subsubsection{2. $H_0$: Students perceived their growth in mathematical knowledge equally in both sections (quiz $=$ control).}

Students were asked in the post-survey of their self-perceived growth in mathematical knowledge. With a $p$-value of $0.151$, the hypothesis was not rejected.

\subsubsection{3. $H_0$: Students perceived their growth in confidence in mathematics equally in both sections (quiz $=$ control).}

Students were asked in the post-survey of their self-perceived growth in confidence in mathematics. With a $p$-value of $0.482$, the hypothesis was not rejected.

\subsection{Focus Group}

During the Focus Group session, students were asked to answer questions on the classroom atmosphere and give broad ideas and opinions of positive and negative things about the course. When asked for three words to describe the course, the students from the control section used words like ``lots of proofs'' and ``solid linear systems'' (in reference to course content), the quiz section students also shared some of these answers but also emphasized ``fun'' and ``[having a] good time.''

Similarly in the answers for the other questions, students in the control section were mostly focused on the intensity of the course and the material, students in the quiz section expressed opinions about the quizzes themselves and the overall lowered stress levels. In part, they discussed how the quizzes as a tool that ``forced [them] to break up'' the material as well as a way to test their learning of the material, knowing that they could retake it without penalty. Others noted the abundance of time there was to take a quiz, compared to its relative brevity, in particular to how midterms are traditionally given in a block of time. Some students mentioned how quizzes could ``pile up if [they] were not careful'' and expressed a desire for a deadline for the first take of the quiz in order to help them not fall behind.

Furthermore, some students from the quiz section raised a question of fairness for their peers and friends from other classes, expressing how it was ``hard to see them fail'' after an exam while not being able to help or relate to them.

\section{Discussion}

The results of the study were showed no significant differences quantitatively in both sentimental factors and academic achievement between the two sections. Qualitatively, however, students experienced a much lower level of stress in the quiz section. Considering these two facts together, I contend that the study ended up having a positive impact on the students overall, since introducing this change could mean introducing better opinions of the learning environment while maintaining the academic rigor. That being said, the sample size and population of the study were not only small but also unique.

\subsection{A (not so) brief note on Mudd}
To understand how self-paced assessments fit into the classroom in this case study, it is critical to note the nature of the college in study as well. The college in this study is Harvey Mudd College, housing about 800 undergraduate students, located an hour from Los Angeles. The college focuses primarily on 6 departments in STEM, where all students are required to complete at least two semesters of coursework in each department, referred to as the Common Core. The introductory linear algebra course in this study, Math 40, is part of the Common Core. All students are therefore expected to take this course regardless of intended major. Moreover, particular traits of this school make the self-paced assessment ideal in achieving desired results in self-regulation.

\subsubsection{The Honor Code}
Mudd, short for Harvey Mudd College, places great importance on its Honor Code, which is maintained by students for students to be responsible for integrity in actions for all academic and non-academic affairs on campus. The Honor Code is not decided by faculty nor administration but created and maintained by the student body and respected by all parties of the College. There are consequences to breaking the Honor Code that are decided by students; often, students at Mudd are willing to take on these consequences via self-reporting incidents that are caused, regardless of intention. In many ways, it is a bridge to securing trust between one another that allows for more freedom and power for students as an active member of the College community.

The Honor Code plays a vital role in the practicality of self-paced assessments. It was expected for students to complete the assessments closed-book and independently. Another expectation was that there would be no discussion of the assessments with other students at any point in the semester so as to avoid benefiting students that may not have had completed them. To trust that students would follow these rules, which are impossible to enforce given the intentional absence of supervision, all parties involved must agree to promise integrity. Thus students need to be able to adhere to the Code, and instructors need to be able to trust that students will do so.

Therefore, I contend that a level of respect towards a code similar to that the Honor Code allows for self-paced assessments to maximize effectiveness. Without this, other measures can be taken to enforce integrity involved in the proposed method, such as having designated proctors or times for students to take assessments supervised, but this asks for further effort and resources that make the method harder to implement.

\subsubsection{Class sizes, college demographics}
The first year class of 227 students from which this sample was taken was composed of 22 percent Asian, 21 percent Latino/Latina, 5 percent African or African-American, 31 percent white and 14 percent multiracial students \citep{hmc}. 52 percent were women, which for a STEM-focused institution, Mudd as ranks one of the highest.

With a class this size, most courses have low student to faculty ratios that enable students access to professors more easily than perhaps at larger institutions. Furthermore, Harvey Mudd is an undergraduate-only institution, meaning there are no graduate students on campus who often take teaching assistant or recitation lecturer positions at larger schools. Professors are generally also expected to be dedicated to teaching as much as they are to research; again, this likely gives students a better chance at accessing a professor for help or advice than students elsewhere, and professors time to spend helping and providing for students directly. For instance, the instructor of the case study done here had to hand-grade and come up with all of the assessments and retries, which is a large time commitment that would easily have been unfeasible with a larger class size.

In addition, the math department faculty at Mudd is far more ethnically diverse than the national statistics shown previously. Out of 15 total tenured and tenure-track professors, 5 are women, 2 are black, and 1 is Latino. Moreover, there are 2 gay professors, and of the 3 post-docs and visiting professors (as of the 2018-2019 school year), there were 2 women of which one was Latina. The faculty at Mudd can be considered outlier to the nation-wide numbers and pose a variable that should not be disregarded when examining the study results.

Thus, it is very likely that there are definitely variables in the demographics of students and the school that may have affected the results of this study and method. This would then likely hinder predicting how effective it is at other institutions of different characteristics in population.

\subsubsection{Mental Health and Wellness}
At Harvey Mudd, awareness and improvement of mental health and wellness is actively advocated for by both academic and residential departments. Mudd is known for its intensity and rigor in curriculum; all students must go through three semesters of Common Core, which consists of 10 or so rigorous classes. During the semester in which most students take Math 40, students are also taking at least four other courses that have demanding amounts of homework and effort.

Not surprisingly, stress is a big factor that affects many students' lives. Whether from workload or academic difficulty, students are often sleep deprived or are burnt out from the consistent loads of problem sets. Math 40 is taught at a relatively fast pace than other classes, being a half-semester course, and homework tends to be focused on exercising many new definitions of terms. Thus, students often feel the need to learn the material more quickly than others, which can cause this course to feel more stressful. Since Math 40 is mandatory to graduate, there is a lot of pressure to do well in the class, whether or not it is a course of their liking.

Given the situation, it is the institution's responsibility to reduce student stress as much as possible. Thus, if there exists a method of learning that reduces overall stress but maintains academic effectiveness, this can be viewed as an overall success, particularly at high-stress institutions like Mudd. The qualitative results of self-paced assessments point towards that this style of learning was less stressful than receiving traditional midterm/final exams. Students who noted that they could avoid the pressure-filled environment of taking a timed exam felt the benefit of the experiment. Other students saw the quizzes as a learning tool rather than an assessment, which also may have lowered stress and negativity. If similar institutions are looking to reduce stress among students and promote good mental health, this may be one way to go about doing so.

\subsection{Self-regulation in action.}
In examining the quotes from the Focus Group sessions, there are definite instances where students were unintentionally applying methods of self-regulation.

One student said, ``I used the first quiz like a learning tool'' because they knew that they could see where they needed improvement and retake the quiz without penalty. This is not only a method of self-monitoring but also self-instruction; the student took upon themselves to assess their knowledge and take advantage of the fact that there was no penalty to trying again. In a broader scope, this is metacognition in action, where self-paced assessments are giving them a chance to understand how they learn and where to improve. This also means that the power is in the students' hands to control what they believe is a correct assessment of their learning, rather than a single examination taken at a time not of their choosing.

Some students pointed out possible areas where they wanted to take charge of what could have been done in the experiment better. One student said, they ``wanted a deadline'' for the first try of a quiz in order to avoid having them pile up. The only deadline to these quizzes were that they had to be completed before the end of the semester. By saying that they wanted more deadlines, the student is asking for what they wish would help them better their learning, particularly in helping them self-monitor their habits in completing assignments. On a similar note, another student pointed out that a ``release date'' for each of the quizzes would have helped them keep track of when they were coming so as to help them prepare for it in advance. This is another mode of self-regulation where the student possibly tries to plan ahead and figures out what they need to do logistically for them to minimize cramming.

As mentioned before, Math 40 is a fast course. This was noted by both sections, but the control section was significantly more demanding of the need for more practice problems (particularly on proofs), while the quiz section focused on whether the quizzes were beneficial to keeping them in track through the course. The control sections' frustrations that there were not enough examples or practice problems seem not to have appeared in the quiz section perhaps because the quizzes themselves acted as a sort of practice. How nothing was said of the lack of understanding on how to do proofs shows that the quizzes were able to perhaps achieve more academically than homework and lectures alone. Furthermore, students in the quiz section were actively noting their progress through the course; while a few students noted that they felt they had forgotten a lot of the first couple lectures, others noted that the quizzes forced them to return to these concepts. Self-assessment involves the realization of how one is doing in mastering or understanding a concept--students in the quiz section (no matter what they thought) had a general idea of how they were doing with the material, therefore exercising further metacognition.

While this was a small sample size, there seemed to be optimistic evidence towards how self-paced assessments could improve and engage metacognitive abilities and reinforce students' power in the classroom.

\subsection{A question to consider}

This case study proposes an open question: can self-paced assessments help to address inequitable learning environments?

I ask this question in hopes to get more efforts of gathering a larger data set to work with, both quantitatively and qualitatively. This study was a small example of self-paced assessments could be used in a mathematics classroom. Though the student sample and school demographics are unique, there is a lot that can be done to implement either a part of the method or change the method to cater the need of other institutions and students.

I think that more experiments done at either Harvey Mudd or similar, small liberal arts colleges could give more insight into how effective this method is, in particular focusing on how it could improve the power dynamics of the classroom setting and improve self-regulatory skills. A different focus could also be to see how this reduces stress related to academics without hindering curricula.

All in all, this study has opened up interesting possibilities to explore in a mechanism that ultimately seemed to help students at Mudd feel lees stress and understand their learning while not impacting their academic learning much.
