\chapter{The case study}
\section{Introduction}

Enter the US College Education: projector screens, chalkboards, individual desks, and syllabi stating exact dates to assignments and exams. Despite the 200 or so liberal arts colleges in the US, the variety of student experience is almost nonexistent. In any of these colleges, large lecture halls are ready to be filled with hundreds of students for them to watch a single professor or instructor. Whether a thousand-person introductory course or a ten-person advanced class, a student is expected to consume the material and spit it out, correctly. This is not to ridicule the efforts of certain colleges that are trying to actively reform education, but still the vast majority has remained stagnant.

The case study was designed as an attempt to show direct effects of making a small change and adding an element of self regulation in a mathematics course. To do so, self-regulated, self-paced assessments were put in place of midterm and final examinations.

\section{A (not so) brief note on Mudd}
To understand how self-paced assessment fit into the classroom in this case study, it is critical to note the nature of the college in study as well. Harvey Mudd College is a small liberal arts college with about 800 undergraduate students, located an hour from Los Angeles. The College focuses primarily on 6 departments in STEM, where all students are required to complete at least two semesters of coursework in each department, referred to as the Common Core. Particular traits of this school make the self-paced assessment ideal in achieving desired results in self-regulation.

\subsubsection{The Honor Code}
Mudd, short for Harvey Mudd College, places great importance on its Honor Code, which is maintained by students for students to be responsible for integrity in actions for all academic and non-academic affairs on campus. The Honor Code is not decided by faculty nor administration but created by the students and respected by all parties of the College. In many ways, it is a bridge to securing trust between one another that allows for more freedom and power for students as an active member of the College community.

As humans, students are not perfect abiders to the Honor Code. Violations are dealt with by students as well, determining consequences and punitive measures under student government. To further reinforce trust that may be lost with such violations, the practice of self-reporting is put in place such that students are able to admit their own fault, instead of being the accused.

This explanation is necessary to understand what ways trust can be placed in students. Of the various privileges that come from securing the Honor Code are take-home assessments, which are, as the name states, regular examinations that are typically handed out for students to take at their own leisure, ``at home.'' Students are expected to take no more time than given, as well as not refer to any external materials if the exam is closed-book. This is all expected under no other supervision than the student's own consciousness.

The math department is no exception, either. Take-home exams are a part of nearly all courses, most professors see the value in not letting time or location be a factor in assessments. With the culture at Mudd, students who were asked to take take-home assessments, multiple times, were trusted to do the right thing.


\section{Method}

This study focused on a mandatory introductory linear algebra course, all 24 students being first year students. There were 8 total sections of the course across 4 different instructors, each who taught 2 of these sections. As a method of control, one of the sections from one instructor underwent the study, while the other section from the same instructor remained unchanged. For ease in distinguishing the two, the changed section will be dubbed the ``quiz section'' and the unchanged section will be the ``control section.''

The 24 students of the quiz section did not have any midterm or final examinations. Instead, students were required to finish a total of 10 small assessments, or quizzes, by the end of the course. All 6 closed-book, closed-notes assessments consisted of one or two questions that pertained to the knowledge of the material, and students were expected to finish each within 15 minutes. All quizzes were self-paced and take-home, meaning students had autonomy over when, where, and how they wished to take these assessments.

Homework assignments remained identical to the other 7 sections, and instruction was similar for the two sections under the same instructor. Students were not given the choice to opt out, other than to switch into a different section. No students in other sections were allowed to switch into the section in study.

Students in both sections were asked to fill out a pre-survey during the first week of the course asking for information like demographics and high school math backgrounds. After the course ended, the students were asked to fill out a second survey and participate in focus groups for qualitative feedback.
