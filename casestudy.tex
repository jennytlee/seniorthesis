\chapter{The case study}
\section{Introduction}

Enter my US College Education: projector screens, chalkboards, individual desks and syllabi stating exact dates to assignments and exams. Despite the 200 or so liberal arts colleges in the US, the variety of student experience is almost nonexistent. In any of these colleges, large lecture halls are ready to be filled with hundreds of students for them to watch a single professor or instructor. Whether a thousand-person introductory course or a ten-person advanced class, a student is expected to consume the material and spit it out, correctly. This is not to ridicule the efforts of certain colleges that are trying to actively reform education, but still the vast majority has remained stagnant.

As mentioned in the earlier chapter, I want to see if there can be positive change in this status quo via self-regulation--specifically, through the use of self-paced assessments. The case study was designed as an attempt to show direct effects of making a small change and adding an element of self regulation in a mathematics course. To do so, self-regulated, self-paced assessments were put in place of midterm and final examinations.

\section{Method}
This study focused on a mandatory introductory linear algebra course, listed as Math 40, offered as a part of the common graduation requirements. Math 40 is typically a 7 week course, and historically assign students about 10 assignments and two exams consisting of one midterm and one final. There were total 49 students involved being first year students, and students were split between 2 sections randomly, taught by the same instructor. One of the sections underwent the study. The other section remained unchanged as a control. For ease in distinguishing the two, I will refer to the test section as the ``quiz section'' and the unchanged section as the ``control section.''

The 24 students of the quiz section did not have any midterm or final examinations. Instead, these students were required to finish a total of 10 small, one or two problem assessments, dubbed quizzes, by the end of the course. All 10 closed-book, closed-notes assessments consisted of questions that pertained to the knowledge of the material that was taught up to the day of release. Students were able to retake these quizzes without penalty, but the ultimate grade of the quiz would be determined by the latest attempt. There were no deadlines to any of these quizzes except for the final deadline at the very end of the semester. Students were also expected to finish each within 15 minutes. In other words, all quizzes were self-paced and take-home, meaning students had autonomy over when, where, and how they wished to take these assessments.

The control section took one midterm and one final exam as traditionally done, appropriately timed around half way and at the end of the semester. Homework assignments remained identical to the other sections, and instruction was similar for the two sections under the same instructor. There were no extra or additional assignments, nor were there fewer assignments, in any one of the sections.

There were six other sections of Math 40 being taught at the same time by other instructors, but other than the same material being taught, had nothing to do with the study. Students were not given the choice to opt out of the study while in either the quiz or control section, but could choose to drop or switch into a different section not part of the study. No students in other sections were allowed to switch into either sections in study.

Students in both sections were asked to fill out a pre-survey during the first week of the course asking for information including demographics, high school math courses, and family backgrounds. This survey also asked for: self-confidence and assessment in mathematic ability, belief in the need for certain elements in creating an intellectual environment, and level of comfort in asking questions.

After the course ended, students were asked to fill out a post-survey, which included questions about students' perceived growth in mathematics ability and confidence. They were also asked to participate in focus groups for qualitative feedback using questions presented by a non-interactive individual not part of the study (neither I nor the instructor were present in the room).

\section{Results}

Overall, the pre- and post-surveys found little statistically significant evidence of differences between the two sections. Qualitatively, there were notable differences in descriptions of the experience in the course; in particular, many students in the quiz section noted a lower level of perceived stress.

\textit{Necessary tables and graphs here; in progress work with LPB.}

\subsection{Focus Group}

During the Focus Group session, students were asked to answer questions on the classroom atmosphere and give broad ideas and opinions of positive and negative things about the course. When asked for three words to describe the course, the students from the control section used words like ``lots of proofs'' and ``solid linear systems'' (in reference to course content), the quiz section students also shared some of these answers but also emphasized ``fun'' and ``[having a] good time.''

Similarly in the answers for the other questions, students in the control section were mostly focused on the intensity of the course and the material, students in the quiz section expressed opinions about the quizzes themselves and the overall lowered stress levels. In part, they discussed how the quizzes as a tool that ``forced [them] to break up'' the material as well as a way to test their learning of the material, knowing that they could retake it without penalty. Others noted the abundance of time there was to take a quiz, compared to its relative brevity, in particular to how midterms are traditionally given in a block of time. Some students mentioned how quizzes could ``pile up if [they] were not careful'' and expressed a desire for a deadline for the first take of the quiz in order to help them not fall behind.

Furthermore, some students from the quiz section raised a question of fairness for their peers and friends from other classes, expressing how it was ``hard to see them fail'' after an exam while not being able to help or relate to them.

\section{Discussion}

The results of the study were showed no significant differences quantitatively in both sentimental factors and academic achievement between the two sections. Qualitatively, however, students experienced a much lower level of stress in the quiz section. Considering these two facts together, I contend that the study ended up having a positive impact on the students overall, since introducing this change could mean introducing better opinions of the learning environment while maintaining the academic rigor. That being said, the sample size and population of the study were not only small but also unique.

\subsection{A (not so) brief note on Mudd}
To understand how self-paced assessments fit into the classroom in this case study, it is critical to note the nature of the college in study as well. The college in this study is a small liberal arts college named Harvey Mudd College with about 800 undergraduate students, located an hour from Los Angeles. The college focuses primarily on 6 departments in STEM, where all students are required to complete at least two semesters of coursework in each department, referred to as the Common Core. Particular traits of this school make the self-paced assessment ideal in achieving desired results in self-regulation.

\subsubsection{The Honor Code}
Mudd, short for Harvey Mudd College, places great importance on its Honor Code, which is maintained by students for students to be responsible for integrity in actions for all academic and non-academic affairs on campus. The Honor Code is not decided by faculty nor administration but created and maintained by the student body and respected by all parties of the College. In many ways, it is a bridge to securing trust between one another that allows for more freedom and power for students as an active member of the College community.
\textit{Add importance of how Honor Code plays into the role of self-paced assessment, note the Code is unique to the College}

\subsubsection{Class sizes, college demographics}
\textit{Add information about the first year class demographics and insight into how this differs from a traditional college}

\subsubsection{Mental Health and Wellness}
\textit{Describe the atmosphere of Mudd and academic rigor}

\subsection{Problems}
\textit{Scalability, instruction burdens}

\subsection{A question to consider}
\textit{Proposal of possible use cases and experiments that can be conducted at other colleges with similar demographics for different courses, student populations.}
