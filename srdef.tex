\chapter{Self-Regulation}

\section{Definition}
This is the definition of self-regulation provided by Zeider, Pintrich and Boekaerts' \emph{Handbook of Self-Regulation}:
\begin{displayquote}
  Self-regulation refers to self-generated thoughts, feelings, and actions that are planned and cyclically adapted to the attainment of personal goals \citep{corte_chapter_2000}.
\end{displayquote}

This can be broken into two parts. First, it focuses on self-generation, indicating the necessity for an individual's own efforts and thus emphasizing power in the self. That way, the generated thoughts and actions can be structured to their own goals and needs, not those of others, such as the society or instructor. The word ``cyclically'' should be underlined here, as it points out how the process can be self-sustained, reinforced by practice and initial support.

Adopting a focus on self-generation of thoughts and actions which lead to attaining personal goals is a statement describing the achievement of power, at the least power over one's self. In a typical classroom setting composed of a single instructor and a group of tens to hundreds of individuals, there exists a power dynamic. The instructor is given an amount of control over the students' actions and knowledge that is only sometimes challenged, and only in some ways.

Thus, promotion of self-regulation will accomplish two parts for students: one in which the process of self-generating their own thoughts and actions will shift the locus of power away from being centralized at the instructor, and two in which the learning experience can be shaped to fit personal needs and goals, instead of generalized versions often presented in traditional classes.

The use of self-regulation in mathematics learning can be a driving force in achieving a change in perspective of mathematics in society, both by institutions and students alike. As I discussed in the previous chapter, it is important to keep in mind that the goal of self-regulation lies in rehumanizing students and bringing more equity in a postsecondary mathematics classroom.

\section{Origin and forms of self-regulation in history.}

To begin looking into how the core ideas of self-regulation came about, mastery-based learning is a fit place to start. Simply put, mastery learning seeks to incorporate individualized pacing of progression through the course material. Developed by Benjamin Bloom, mastery-based learning expects students to have complete or near-complete mastery of concepts before proceeding further into the material \citep{bloom_learning_1968}. Over the years mastery learning has taken on many variations, some more successful than others, but all focus on individual pacing and developing autonomy in a student's ability to learn \citep{bradley_evaluating_2017}. The biggest takeaway from studies done in mastery learning is the positive impact it has on students despite how much it differs from traditional methods of teaching \citep{zollinger_impact_2017}.

As shown with the many studies done of the impact of mastery learning, self-regulation departs from traditional instruction. Thus, how or why self-regulation should be a part of school likely does not come to most educators immediately. For a large part, if not all, of a young student's life in academics, the classroom is where they are instructed to do one thing or another. Report cards and other assessments and evaluations are the only sources of feedback.

For college and postsecondary education where classroom sizes go upwards to hundreds and even thousands of students, the feedback given to students is difficult to refute or debate, especially when individual attention is hard to receive. Moreover, it could be that chances to improve one's grades are really given only once or twice a semester after midterm grades are posted.

This problem arises because mathematics education is rarely in the form different from the lecture-recitation style classes. Seminar or discussion-heavy classes in mathematics are generally unheard of, let alone calling on students for participation aside from asking for answers. With bigger classes, asking questions in itself becomes a challenge, often perceived as being a waste of lecture time; practically impossible if the lecturer spares zero opportunities for questions. Truly, the conversation is one-sided, with little or no reception from the students' in their understanding.

In this status quo, it is unthinkable to ``personalize'' a course to meet a particular student's needs. More so, students have few chances to champion for themselves what they were lacking in the education they received. It is hardly reasonable to claim that one form of learning is the best way for every student to achieve success, as will be discussed in further detail below.

Looking only in terms of providing individual attention for academic achievement, attempts so far include remedial classes. Unfortunately, these often further reduce self-efficacy in underachieving students, as the students are singled out and required to take these extra classes under the description that they are struggling or behind, increasing both physical and mental stress factors \citep{martin_developmental_2017}.

Nevertheless, self-regulation takes many different forms and can be adapted to any type of classroom. In both methodology and focus, self-regulation can be incorporated at small or large scales. Detailed below are some (but certainly not all) ways in which self-regulation can take place in instruction \citep{montague_self-regulation_2007}. In addition, self-regulative strategies will often encompass a mix or overlap of the listed forms, thus none are mutually exclusive of another.

\section{Self-Assessment and Evaluation}
Self-assessment and evaluation can be pertinent to either qualitative evaluations of cognitive skills related to work ethic and habits or quantitative assessment of knowledge on concepts.
The goal of some self-assessment and evaluation methods revolves around helping students practice independent realization of their own necessities and strengths in learning, and hence increase self-efficacy as well as a feeling of empowerment.

Evaluating work practices in mathematics can be achieved through a variety of ways, including worksheets that ask students to outline how they solved certain problems, reflection assignments that encourage students to evaluate their own weaknesses and strengths, and checkboxes to ensure certain practices were done \citep{montague_self-regulation_2007}. Such metacognitive processes can help students find and understand for themselves where they can improve in a way that doesn't explicitly expose particular weaknesses to their peers or instructors.

In recent years, self-assessment of course material and knowledge recollection is sometimes found in form of online-based classrooms, which reduces the work load of instructors to grade and follow through with each individual's assessments, as well as prevent academic dishonesty \citep{ventista_self-assessment_2018}. However, the nature of online based learning is that a computer and a reliable internet connection is a luxury that students should not be expected to have, especially when equitable practices are in concern, as mentioned before.

A section below outlines more specifically the kinds of self-assessments that can accurately aid student learning and provide ways the reduce unequal power dynamics. Moreover, the case study found in this paper describes one specific example of self-regulation which seeks to implement a fair way to provide student autonomy by encouraging self-assessment of skills and improving self-efficacy in college mathematics. Once again, the goal of any method should be to increase empowerment of students, reduce inequitable practices, and improve the student experience. Thus, it is key to think about the benefits and fallbacks of everything discussed below.

\subsection{Self-Instruction}
Self-instruction looks into empowering students to learn the material on their own, thereby also instilling the belief that they are capable. Naturally, there is some risk associated to self-instruction, and therefore is often paired with supplementary activities or practices that solidify or clarify learning.

Examples of self-instruction cross an entire spectrum of student independence in the classroom, from full-autonomy where students decide what should be covered and how, to partial-autonomy that expects students to learn the material provided by an instructor \citep{burris_developmental_1972}.

Recently, self-instruction in mathematics has taken form via flipped classrooms, in which the learning of material is done outside of the scheduled class time through slides and recorded lectures \citep{lage_inverting_2000}. This type of instruction reserves space and time for students to spend class time on group activities and more in-depth discussions of mathematics beyond the surface level of concepts, but also increases responsibility on the students to learn the material correctly on their own.

\subsection{Self-Monitoring}
Self-monitoring is similar in nature to the other self-regulation forms, but focuses more on providing immediate feedback. In mathematics, a checklist of commonly found errors are provided for students to check intermediate steps while solving problems \citep{dunlap_self-monitoring_1989}. The checklist is subsequently personalized for each student as mistakes are made, and eventually they were removed as a form of assistance. Results from some studies showed an increase in achievement \citep{dunlap_self-monitoring_1989}.

There are obvious challenges with this form of self-regulation, as it  poses massive workloads realistically impossible for some teachers or instructors. Furthermore, such a checklist is often difficult to formulate for mathematics classes above introductory, more computational courses. It is important, still, to see the benefits of introducing students to metacognitive methods such as creating a checklist on their own to aid their learning.

\section{An example of self-regulation in the classroom: using self-paced assessments.}
While all of the various ways self-regulation that takes place in the classroom has benefits, self-assessment of course material has tangible and scalable opportunities that touch upon self-instruction and self-monitoring as well. More specifically, self-paced assessment allows for the students to take control of the pace they are expected to assess their learning the material. A more explicit example of how this variation of self-regulation in action can be implemented is explored in the case study later.

The methodology behind how self-paced assessments can be used is as simple as its name sounds. In the study, all assessments are conducted by the students on their own time and in their own choice of setting. Of the many stressful factors students are exposed to in college, examinations are one of the most prominent sources of stress \citep{abouserie_sources_1994}. Students are expected to cover a large amount of the course material and regurgitate it coherently within a set amount of time. In a traditional setting, all students in the course are asked to have the material digested by the time the exam is given to a level where basic concepts can be extended to applications. There is no chance or way to show that improvements can be made after exams are taken--in other words, a one-time assessment is the determining factor of a student's understanding of the material.

Described in this way, it sounds naive to trust that traditional methods of assessment and instruction are fair and accurate ways to judge the complex and multidimensional understanding of material students can have. In addition, since examinations often act as a tool to give grades, which means questions that demand the creative process (such as open questions in the field) are likely left out to avoid vague, subjective grading. Despite how critical creative thinking is for mathematical research and exploration, if exams avoid asking such questions, students aren't able to practice necessary skills for furthering knowledge.

Self-paced assessment seeks to remedy some of many issues with traditional methods of teaching. For example, instead of one large assessment instrument that covers weeks to months of material, multiple smaller assessments will ultimately achieve the same goal of checking the state of students' understanding while entirely removing the stressful factor of having to review and cram large amounts of material at once.

Second, students are relieved of the burden of having understood everything on a strict schedule. Individual styles and paces of learning is entirely ignored in the status quo, despite just how vastly spread out these can be \citep{busato_intellectual_2000}. The only expectation is that students are to complete the set of assessments by the end of the course. It is expected that the assessments would be handed out on a timely manner when the material being assessed is covered, but it is not expected that the student would be prepared at that moment to be tested on it. Having the independence to be able to take the assessments at their own pace is essentially how self-regulation takes a role here.

Third, students will have a chance to retake these assessments if they feel as though they were not sufficiently prepared or think that they did not fully comprehend the material upon taking the assessment. Penalizing students who simply made an algebra mistake or could not finish an assignment that covered an important concept are simply unfortunate events that should not be deterministic of a student's achievement in the course. Rather, it should be encouraged for students to self-evaluate and test where they are in the course and use the retake opportunity to their advantage to figure out where they are lacking and where they are strong. This not only reduces time spent on reviewing material a student may be already strong on, but also creates efficient study habits that builds metacognition.

Self-paced assessment as described here relies heavily on trust between students and teachers. In pop culture students are compared to prisoners, both groups of individuals under complete control, following a rulebook of a system set in stone. Students from a young age are praised for following directions and punished for acting out \citep{inbar_free_1996}. Eventually those who ``succeed'' in school are those students who were most obedient and studied what was given to them, without question. The snowball effect goes the other way as well, in which incriminating or humiliating students for some actions and grades lead to building further negative associations to school, reducing their desire to learn or participate. This phenomenon aligns with the school-to-prison-pipeline metaphor, which may further perpetuate negativity \citep{crawley_examining_2018}.

Thus, giving students control over their own learning is essentially an action of giving students trust. Construction of trust in each other can work to flatten the strict hierarchy that exists today. Particularly for higher level postsecondary institutions, students are imminent members of academia and society at a level of maturity that deserves trust, and subsequently, equity in power in the classroom. Trusting that students can be responsible for their own learning in these assessments leaves greater individual impact that in turn affects how society views education.
