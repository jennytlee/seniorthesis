\chapter{Math is not fair.}

The title of this chapter may be confusing. What do I mean when I discuss fairness in mathematics?

I grew up hearing and thinking that math was neutral; that unlike in literature or social studies, the instructions told you $1+1$ was always $2$, no matter who you were or what you believed in.

As Friere discusses, the role a student typically takes in a traditional classroom is to act as a container that simply accepts information and regurgitates out appropriately 
\citep{freire_pedagogy_2014}. The system disregards fostering the capabilities of a student to process and apply independent thought, and the instructor is not expected to have their students be able to do so, either.

A student, however, is not a memorizing machine. There is always additional context, both social and historical, outside of the classroom that definitely affects the way a student perceives and performs in mathematics, particularly in the college or university level. With biases and obstructions that exist in ways to curb student achievement, intentionally or not, mathematics education is not fair in various ways that I'd like to look at different levels of society.

\section{Implicit Biases by Instructors}
The one individual that interacts and directly impacts every student in a classroom is the instructor. Hence, if an instructor were to hold preemptive opinions or biases that pertain to particular students, no matter how subconsciously, this may impact their preferences for or against certain students and their performances \citep{green_implicit_2007}. Specifically, this proposition comes from the Harvard implicit association study, which essentially showed how people could hold biases or preferences without explicitly portraying them using a test that detected differences in reaction speeds to making adjective-subject associations.

Using this test, a study done in 2015 showed that male scientists had a tendency to associate science with males more than female scientists \citep{smyth_genderscience_2015}. In fact, this discrepancy was the greatest with associating males with engineering and mathematics, exceeding 0.8 standard deviations. This then possibly implies that male professors are likely to hold a stronger association between male students and mathematics. Referring back to Table \ref{table:gender}, where the percentage of full time female faculty members was less than that of male members, this further suggests bias that a female student may be subjected to during her career.

In fact, a study conducted in 2012 showed that science faculty, regardless of gender and race, preferred male students over female students \citep{moss-racusin_science_2012}. The study involved using two nearly identical fictional individuals and seeing who was more likely to be hired as a laboratory manager. The only difference was their implied gender, deduced by the names ``John'' and ``Jennifer.'' Despite all other attributes being identical, there was significant preference for hiring John; participants scored him higher than Jennifer on all marks on average.

These studies present a strong case for how instructors may hold presumed assumptions towards certain students in mathematics. Certainly what this implies is that if this were true, students subject to negative biases must work harder than their counterparts to impress or succeed and face challenges that make it increasingly difficult remain in the field despite potential opinions being held against them.

For instance, if women are already minoritized in the status quo, it manifests a vicious cycle that likely feeds biases against them. Called ``stereotype threat,'' consequences can result in lasting effects on the environment for women in mathematics \citep{spencer_stereotype_1999}. In tests, women performed substantially worse in tests when told that there were differences in achievement based on gender, compared to equally achieving men. Other experiments handling racial differences showed similar results; African-American subjects who were realized of how their race would perform prior to taking a test were more likely to perform worse than white subjects \citep{steele_stereotype_1995}. Procedures in both experiments included both explicit (direct, verbal delivery) and implicit (diagnostic surveys that served as stereotype-activation) ways of priming subjects of their identities and related stereotypes. Regardless of procedure, subjects who were threatened by stereotypes performed objectively worse. This points to the possibility of how instructors imposing certain biases can actively impact students' performances.

Another important example of stereotyping and implicit biases taking action is actions of microaggression, which describes any seemingly small behavior, including unvocalized assumptions, that relays hostility or prejudiced views towards a marginalized group, unintentional or not \citep{sue_racial_2007}. When unnoticed or ignored, microaggressions towards ethnic minority groups feed racism, fueling a mindset that only continues to be confirmed as a correct one. As a result, impacted students fall further into the mindset of feeling less capable in the classroom.

\section{Structural Biases in Institutions}
To begin discussing how postsecondary institutions present biases against some students and privileges to others, an understanding of how students are filtered into these institutions in the first place must be established. For a typical high school student in the US, about 95\% of college applications require SAT or ACT scores, which imply that this score is often used as a metric to determine admission into the school \citep{morse_about_2008}.

The problems associated with using this score have been firmly studied and confirmed in research; in particular, studies show how it tends to hold advantages for already privileged students \citep{buchmann_shadow_2010}. In mathematics, women and black students tend to score lower than male and white students, which also in turn pose more stereotype threats towards these students \citep{lovaglia_stereotype_2004}. This means therefore that students admitted into notable colleges were subject to institutional biases before even setting foot on campus, which then affects their opportunities further down the road.

Even if students were to be admitted into such schools, they may face roadblocks in accessing or utilizing certain resources at the institution. More often then not, students are expected to be have financial resources that enables them to have not only important academic materials like textbooks and computers, but also life essentials like meals. Many students experience stress and pressure to afford certain amenities or at least have ways in which they can provide the necessary costs, be it through work or loans \citep{ross_sources_1999}.

This connects to a bigger problem in the context of racial identities and socioeconomic groups. The 2007-2011 census provides enough quantitative evidence of the existing, unequal distribution of poverty among different races. American Indians and African Americans came in the highest at about 26\% of the population being in poverty, more than a double in comparison to the 11.6\% of whites \citep{bureau_poverty_2013}. A 2018 New York Times article showcasing a study conducted on white and black men showed that of the 5,000 white and 5,000 black boys who grew up in poverty, 48\% of black boys grew up to remain in poverty and only 2\% grew to be rich, while 31\% of white boys remaining in poverty and 10\% became rich \citep{badger_extensive_2018}.

Objectively, this means it is likely there are more non-white students thinking about and suffering from financial difficulties than their white counterparts. Stress from life-related sources negatively impacts students' achievements in school \citep{andrews_relation_2004}. Thus, this gives reason to believe that students with financial difficulties are probably more prone to experiencing academic difficulties than those without. In 2013, enrollment percentages in postsecondary education showed about a near 10\% difference between white (42\%) and black (34\%) students \citep{musu-gillette_status_nodate}. Graduation rates were similar, lowest for black students at around 41\%. These numbers dip down further for STEM (Science, Technology, Engineering, Mathematics) degrees, with about 11\% fewer black students graduating.

The 2015 study conducted by the Conference Board of Mathematical Sciences showed that 71\% of full time mathematics professors at PhD awarding universities were white, compared to 1\% black and 4\% Hispanic. In addition, 22\% of total professors were women, of which 16\% were white; approximately 0\% were black and 1\% Hispanic.

What this means for students of color sitting on the other side of the podium is a definite disparity in the number of professors that share their racial background. This lack of having a proper role model impacts the belief a student has that they can succeed, otherwise known as self-efficacy \citep{thevenin_mentors_2007}. With lowered self-efficacy comes lowered achievement, unsurprisingly \citep{motlagh_relationship_2011}. Societal and economical factors that relate to academic achievement in these ways means they cannot be ignored when considering equitable practices in mathematics education.

\section{Cultural Obstructions}
When stating that there are cultural obstructions that contribute to inequity in mathematics education, it is not to say that the action of finding a derivative is somehow racially charged or unfair to a specific group of people. Rather, how institutions teach a mathematical concept and the myriad of assumptions made in the process shape the role mathematics takes in classrooms and eventually our society.

\subsection{There is oppression.}
Specifically, I contend that current practices in mathematics is biased against some and privileged against others. In his paper on anti-oppressive education, Kevin Kumashiro outlines ways in which oppression exists in classrooms today that works against ``the Other,'' referring to traditionally marginalized in society, and discusses how to bring about anti-oppressive approaches to education, examining strengths and weaknesses of each \citep{kumashiro_toward_2000}.

Kumashiro categorizes the approaches to anti-oppressive education, including understanding the need for education for the Other, the need for education about the Other, education that is critical of privileging and othering, and education that works to bring change in students and society. He points out that oppression in all four of these flavors comes from the underlying belief that ``normal'' equates to cis-gendered, heterosexual white men \citep{kumashiro_toward_2000}. Furthermore, a flawed and misleading understanding of the Other perpetuated by stereotypes add to further setting normalcy away from the Othered. In this way, he points to how whiteness has served in projecting mathematics as a ``neutral'' subject. As Rochelle Gutierrez writes on how mathematics assumes to have no ``color'' or cultural associations:
\begin{displayquote}
  {[In]} many mathematics classrooms, students are expected to leave their emotions, their bodies, their cultures, and their values outside the classroom walls, stripping them of a sense of wholeness \citep{gutierrez_embracing_2012}.
\end{displayquote}

\subsection{Mathematics education is a racial project.}
Mathematics is historically not led uniquely by white Europeans. Prominent advancements were made by individuals from all over world. But when I ask fellow math majors for names of famous mathematicians, what I hear are not Srinivasa Ramanujan, Hypatia, or Dorothy Vaughn; but Euler, Pythagoras and Fermat. The problem in question lies exactly here--whiteness is rarely questioned in this context of mathematics. Perhaps then a valid question to ask is, why has mathematics education been so predominantly white?

Danny Martin attempts to answer the question by discussing the deliberate and prevailing racial agendas that use mathematics education as a tool aligned with sustaining a white framework in society, particularly in attaining market-oriented goals \citep{martin_race_2013}. He describes beyond what statistics show of underrepresentation in mathematics by analyzing literature that seems to promote mathematics education reform but also contributes to the agendas by not going into detail the key position race and racialization takes in the picture.

Furthermore, Martin argues that mathematics education itself has aligned with these agendas in keeping whiteness in control because it has been able to stay away and immune from possibilities of being racially charged or placed under racial politics. The way in which this was possible, he describes, is the existence of white institutional spaces that conform to particular characteristics:

\begin{displayquote}
  (a) numerical domination by Whites and the exclusion of people of color from positions of power in institutional contexts, (b) the development of a White frame that organizes the logic of the institution or discipline, (c) the historical construction of curricular models based upon the thinking of White elites, and (d) the assertion of knowledge production as neutral and impartial, unconnected to power relations \citep{martin_race_2013}.
\end{displayquote}

Many points made previously directly support how white institutional spaces have existed and continue to exist in society today; domination in number by white professors, curricular concepts based on and named off of historical white mathematicians, and implications that there are no relations between mathematics in societal and racial constructs. If these four points do manifest in reality, Martin's claims suggest the deep-rooted locus of power residing in whiteness and calls for the need to critically analyze institutions.

While these points of indication are clear, they are broad. Thus in fulfilling the need for specificity and examples, Dan Battey and Luis Leyva describe these characteristic indicators of such spaces by breaking down into specific areas:
\begin{table}[htb]
  \begin{center}
    \begin{tabular}{l | l}
      Dimension & Elements\\
      \hline
      \multirow{4}*{Institutional} & Ideological Discourses\\
      \cline{2-2}
      & Physical Space\\
      \cline{2-2}
      & History\\
      \cline{2-2}
      & Organizational Logic\\
      \hline
      \multirow{3}*{Labor} & Cognition\\
      \cline{2-2}
      & Emotion\\
      \cline{2-2}
      & Behavior\\
      \hline
      \multirow{3}*{Identity} & Academic (De)Legitimization\\
      \cline{2-2}
      & Co-construction of Meaning\\
      \cline{2-2}
      & Agency and Resistance
    \end{tabular}
  \end{center}
  \caption{Framework of Whiteness in Mathematics Education. A more detailed breakdown of each dimension is detailed in their paper \citep{battey_framework_2016}.}
  \label{table:framework}
\end{table}

This table outlines a way in which institutions can check whether whiteness is being perpetuated, and how so. But understanding how mathematics has continued to be used as a tool of racial and ethnic justice can be achieved through an example. In particular, Nicole Joseph uses these characteristics to identify and recognize how white institutional spaces influence the mathematics learning of black women \citep{joseph_black_2017}. Martin's notion of white institutional spaces come from critical race theory, which stems from legal backgrounds, serves to present a model that recognizes inequities in race and challenge existing predominant ideologies \citep{solorzano_critical_2000}. Joseph describes how critical race theory permits the claim that mathematical spaces are not neutral. Her identification of how mathematics education has been unfair toward black women presents a concrete way in which individual students and their educations are affected under white institutional spaces.

\subsection{It is possible to change.}
I've thus far presented a problem in which inequitable practices in mathematics education are real and impact students. Remodeling education to eliminate inequitable practices in mathematics seems a daunting task for any one single institution, let alone an entire nation, to tackle. For all four educational approaches presented in his article, Kumashiro argues that there needs a prominent desire for change for any change to even occur \citep{kumashiro_toward_2000}. As it is with any kind of change, there is resistance from the presiding body of power.

But Rochelle Gutierrez tells us there is hope. In her quote earlier she describes how students are ``stripped away'' of their sense of wholeness in the mathematics classroom, hence addressing an issue of how students are dehumanized objects in the classroom \citep{gutierrez_living_2017}. In discussing possible ways to ``rehumanize'' students, she emphasizes the necessity to recognize hierarchies in classrooms and shifting the role of authority \citep{gutierrez_why_2018}.

More broadly, Gutierrez's ideas are grounded in her theory of two axes that cross in describing dimensions of equity: Access and Achievement make up the dominant axis, while Identity and Power make up the critical axis \citep{gutierrez_2009}. The dominant axis describes what would determine a student's ability in mathematics, while the critical axis describe what would measure a student's ability to think critically of mathematics, perhaps bringing change. Gutierrez contends all four are needed to build an equitable mathematics learning environment. In particular, she describes the Access dimension being a ``precursor'' to Achievement and Identity to Power, which implies the two former need to exist and establish before the latter.

Thus, her proposal that moving the locus of power is one way in which the critical axis is being used--she brings the students' identities into the conversation. In the next chapter, we explore a way in which this can be done via self-regulation.
