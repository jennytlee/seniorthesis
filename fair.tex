\chapter{In context, math is not fair.}

The title of this chapter may be confusing. What do I mean when I discuss fairness in mathematics?

Far too often, I grew up hearing and thinking that math was neutral; that unlike in literature or social studies, the instructions told you $1+1$ was always $2$, no matter who you were or what you believed in.

As Friere discusses, the role a student typically takes in a traditional classroom is to act as a container that simply accepts information and regurgitates out appropriately (Friere, 1968). The system disregards fostering the capabilities of a student to process and apply independent thought, and the instructor is not expected to have their students be able to do so, either.

A student, however, is not a memorizing machine. There is always context outside of the paper that definitely affects the way a student perceives and performs in mathematics, particularly in the college or university level. With biases and implicit obstructions that exist in ways to curb student achievement, mathematics education is not fair in various ways that I'd like to look at different levels of society.

\section{Implicit Biases by Instructors}
If there is one individual that interacts and directly impacts a student, it is the instructor of a class. Hence, if an instructor were to hold preemptive opinions or biases that pertain to particular students, no matter how subconsciously, this may impact their preferences for or against certain students and their performances. Specifically, this proposition comes from the Harvard implicit association study, which essentially showed how people could hold biases or preferences without explicitly portraying them using a test that detected differences in reaction speeds to making  adjective-subject associations.

Using this test, a study done in 2015 showed that male scientists had a tendency to associate science with males more than female scientists (Smyth and Nosek, 2015). In fact, this discrepancy was the greatest with associating males with engineering and mathematics, exceeding 0.8 standard deviations. This then possibly implies that male professors are likely to hold a stronger association between male students and mathematics. Referring back to Table \ref{table:gender}, where the percentage of full time female faculty members was less than that of male members, this further suggests bias that a female student may be subjected to during her career.

In fact, a study conducted in 2012 showed that science faculty, regardless of gender and race, preferred male students over female students (Moss-Racusin, et al., 2015). The study involved using two nearly identical fictional individuals and seeing who was more likely to be hired as a laboratory manager. The only difference was their implied gender, deduced by the names ``John'' and ``Jennifer.'' Despite all other attributes being identical, there was significant preference for hiring John; participants scored him higher than Jennifer on all marks on average.

These studies present a strong case for how instructors may hold presumed assumptions towards certain students in mathematics. Certainly what this implies is that if this were true, students subject to negative biases must work harder than their counterparts to impress or succeed and must have more discipline to remain in the field despite possible opinions being held against them.

Furthermore, if female students are already minoritized in the status quo, it manifests a vicious cycle that likely feeds biases against them, which in turn makes it increasingly difficult for such students to remain in the field. Called ``stereotype threat,'' consequences can result in lasting effects on the environment for women in mathematics (Spencer et al., 1999).

An important example of stereotyping and implicit biases taking action is actions of microaggression, which describes any seemingly small behavior, including unvocalized assumptions, that relays hostility or prejudiced views towards a marginalized group, unintentional or not. When unnoticed or ignored, microaggressions towards ethnic minority groups feed racism, fueling a mindset that only continues to be confirmed as a correct one. As a result, impacted students fall further into the mindset of feeling less capable in the classroom.

\section{Structural Biases by Institutions}
To begin discussing how postsecondary institutions present biases against students, an understanding of how students are filtered into these institutions in the first place must be established. For a typical high school student in the US, a standard college application requires SAT or ACT scores that often are used as a metric to determine admission into the school.

The problems associated with using this score have been analyzed again and again; in particular, studies show how it tends to hold advantages for already privileged students (Buchmann et al., 2010). In mathematics, female and black students tended to score lower than male and white students, which also in turn pose more stereotype threats towards these students (Jencks, 1998). This means therefore that students admitted into notable colleges were subject to institutional biases before even setting foot on campus, which then affect their opportunities further down the road.

Even if students were to be admitted into schools, they may face roadblocks and unfairly provided opportunities at the institution. Far often then not, students are expected to be at some financial standing that enable them to have not only resources like textbooks and computers, but also essentials like meals. There exists an assumption that students can afford certain amenities or at least have ways in which they can provide the necessary costs, be it through work or loans.

This presents a bigger problem between racial identities and socioeconomic differences. The 2007-2011 census provides enough quantitative evidence of the existing, unequal distribution of poverty among different races. American Indians and African Americans came in the highest at about 26\% of the population being in poverty, more than a double in comparison to the 11.6\% of whites (Macartney, 2013). A 2018 New York Times article showcasing a megastudy conducted on white and black men showed that of the 5,000 white and 5,000 black boys who grew up in poverty, 48\% of black boys grew up to remain in poverty and only 2\% grew to be rich, while 31\% of white boys remaining in poverty and 10\% became rich (Badger, 2018).

Objectively, this means it is likely there are more non-white students thinking about and suffering from financial difficulties than their white counterparts. It's simply not fair to expect the same degree of achievement from students that need not think about running out of money by the end of the week to those that do. In 2013, enrollment percentages in postsecondary education showed about a near 10\% difference between white (42\%) and black (34\%) students (Musu-Gillette, 2016). Graduation rates were similar, lowest for black students at around 41\%. These numbers dip down further for STEM (Science, Technology, Engineering, Mathematics) degrees, about 11\% apart.

The 2015 study conducted by the Conference Board of Mathematical Sciences showed that 71\% of full time mathematics professors at PhD awarding universities were white, compared to 1\% black and 4\% Hispanic. 22\% of total professors were women, of which 16\% were white; approximately 0\% were black and 1\% Hispanic.

What this means for students of color sitting on the other side of the podium is a definite disparity in the number of professors that share their racial background. This lack of having a proper role model impacts the belief a student has that they can succeed, otherwise known as self-efficacy (Thevenin, 2007). With lowered self-efficacy comes lowered achievement, unsurprisingly (Motlagh, 2011). Once again, we see how the question of how these unfair societal norms factor into reducing the quality of education or effectiveness of education a student receives is rarely raised.

\section{Cultural Obstructions}
When stating that there are cultural obstructions that exist to contribute inequity in mathematics education, it is not to say that the action of finding a derivative is somehow racially charged or unfair to a specific group of people. Rather, the way institutions teach a mathematical concept and the myriad of assumptions made in the process shape the role mathematics takes in our society.

\subsection{There is oppression.}
Specifically, I contend that current practices in mathematics is biased against some and privileged against others. In his paper on anti-oppressive education, Kevin Kumashiro outlines ways in which oppression exists in classrooms today that works against ``the Other,'' referring to traditionally marginalized in society, and discusses how to bring about anti-oppressive approaches to education, examining strengths and weaknesses of each (Kumashiro, 2000).

What Kumashiro discusses includes understanding the need for education for and about the Other, education that is critical of privileging and othering, and education that works to bring change in students and society. He points out that oppression in all four of these flavors comes from the underlying belief that ``normal'' equates to cis-gendered, heterosexual white men. Furthermore, a flawed and misleading understanding of the Other perpetuated by stereotypes add to further setting normalcy away from the Othered. In this way, he points to how whiteness has served in projecting mathematics as a ``neutral'' subject. As Rochelle Gutierrez writes on how mathematics assumes to have no ``color'' or cultural associations:
\begin{displayquote}
  {[In]} many mathematics classrooms, students are expected to leave their emotions, their bodies, their cultures, and their values outside the classroom walls, stripping them of a sense of wholeness (Gutierrez, 2012).
\end{displayquote}

\subsection{This is a racial project.}
Mathematics is historically not led uniquely by white Europeans. Prominent advancements were made by individuals from all over world. But when I ask fellow math majors for names of famous mathematicians, what I hear are not Srinivasa Ramanujan, Hypatia, or Dorothy Vaughn; but Euler, Pythagoras and Fermat. The problem in question lies exactly here--whiteness is rarely questioned in this context of mathematics. Perhaps then a valid question to ask is, why has mathematics education been so predominantly white?

Danny Martin attempts to answer the question by discussing the deliberate and prevailing racial agendas that use mathematics education as a tool aligned with sustaining a white framework in society, particularly in attaining market-oriented goals (Martin 2012). He describes beyond what statistics describe of underrepresentation in mathematics; analyzing literature that is meant to promote mathematics education reform, including Friere's \textit{Pedagogy of the Oppressed}, stays vague and removed from describing race and racialization in the picture.

Furthermore, he argues that mathematics education has aligned in keeping whiteness in control because it has been able to stay away and immune from possibilities of being racially charged or placed under racial politics. The way in which this was possible, he describes, is the existence of white institutional spaces that conform to particular characteristics. Describe in detail by Dan Battey and Luis Leyva, these characteristic indicators of such spaces are as follows:
\begin{table}[htb]
  \begin{center}
    \begin{tabular}{l | l}
      Dimension & Elements\\
      \hline
      \multirow{4}*{Institutional} & Ideological Discourses\\
      \cline{2-2}
      & Physical Space\\
      \cline{2-2}
      & History\\
      \cline{2-2}
      & Organizational Logic\\
      \hline
      \multirow{3}*{Labor} & Cognition\\
      \cline{2-2}
      & Emotion\\
      \cline{2-2}
      & Behavior\\
      \hline
      \multirow{3}*{Identity} & Academic (De)Legitimization\\
      \cline{2-2}
      & Co-construction of Meaning\\
      \cline{2-2}
      & Agency and Resistance
    \end{tabular}
  \end{center}
  \caption{Framework of Whiteness in Mathematics Education (Battey and Leyva, 2016)}
  \label{table:framework}
\end{table}

\subsection{It is possible to change.}
In this light, remodeling education to eliminate inequitable practices in mathematics seems a daunting task for any one nation, let alone an institution, to tackle. For all four educational approaches presented in his article, Kumashiro argues that there needs a prominent desire for change for any change to even occur (Kumashiro, 2000).
