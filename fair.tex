\chapter{In context, math is not fair.}

The title of this chapter may be confusing. What do I mean when I discuss fairness in mathematics?

Far too often, I grew up hearing and thinking that math was neutral; that unlike in literature or social studies, the instructions told you $1+1$ was always $2$, no matter who you were or what you believed in.

As Friere discusses, the role a student typically takes in a traditional classroom is to act as a container that simply accepts information and regurgitates out appropriately (Friere, 1968). The system disregards fostering the capabilities of a student to process and apply independent thought, and the instructor is not expected to have their students be able to do so, either.

A student, however, is not a memorizing machine. There is always context outside of the paper that definitely affects the way a student perceives and performs in mathematics, particularly in the college or university level. With biases and implicit obstructions that exist in ways to curb student achievement, mathematics education is not fair in various ways that I'd like to look at different levels of society.

\section{Implicit Biases by Instructors}
If there is one individual that interacts and directly impacts a student, it is the instructor of a class. Hence, if an instructor were to hold preemptive opinions or biases that pertain to particular students, no matter how subconsciously, this may impact their preferences for or against certain students and their performances. Specifically, this proposition comes from the Harvard implicit association study, which essentially showed how people could hold biases or preferences without explicitly portraying them using a test that detected differences in reaction speeds to making  adjective-subject associations.

Using this test, a study done in 2015 showed that male scientists had a tendency to associate science with males more than female scientists (Smyth, Nosek, 2015). In fact, this discrepancy was the greatest with associating males with engineering and mathematics, exceeding 0.8 standard deviations. This then possibly implies that male professors are likely to hold a stronger association between male students and mathematics. Referring back to Table \ref{table:gender}, where the percentage of full time female faculty members was less than that of male members, this further suggests bias that a female student may be subjected to during her career.

In fact, a study conducted in 2012 showed that science faculty, regardless of gender and race, preferred male students over female students (Moss-Racusin, et. al, 2015). The study involved using two nearly identical fictional individuals and seeing who was more likely to be hired as a laboratory manager. The only difference was their implied gender, deduced by the names ``John'' and ``Jennifer.'' Despite all other attributes being identical, there was significant preference for hiring John; participants scored him higher than Jennifer on all marks on average.

These studies present a strong case for how instructors may hold presumed assumptions towards certain students in mathematics. Certainly what this implies is that if this were true, students subject to negative biases must work harder than their counterparts to impress or succeed and must have more discipline to remain in the field despite possible opinions being held against them.

Furthermore, if female students are already minoritized in the status quo, it manifests a vicious cycle that likely feeds biases against them, which in turn makes it increasingly difficult for such students to remain in the field. Called ``stereotype threat,'' consequences can result in lasting effects on the environment for women in mathematics (Spencer, Steele, Queen, 1999).

\section{Structural Biases by Institutions}
To begin discussing how postsecondary institutions present biases against students, an understanding of how students are filtered into these institutions in the first place must be established. For a typical high school student in the US, a standard college application requires SAT or ACT scores that often are used as a metric to determine admission into the school.

The problems associated with using this score have been analyzed again and again; in particular, studies show how it tends to hold advantages for already privileged students (Buchmann, Condron, Roscigno, 2010). In mathematics, female and black students tended to score lower than male and white students, which also in turn pose more stereotype threats towards these students (Jencks, 1998). This means therefore that students admitted into notable colleges were subject to institutional biases before even setting foot on campus, which then affect their opportunities further down the road.

Even if students were to be admitted into schools, they may face roadblocks and unfairly provided opportunities at the institution. Far often than not, students are expected to be at some financial standing that enable them to have not only resources like textbooks and computers, but also essentials like meals. There exists an assumption that students can afford certain amenities or at least have ways in which they can provide the necessary costs, be it through work or loans.
\section{Cultural Obstructions}
