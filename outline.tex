\chapter{Outline}
\begin{enumerate}
  \item {\bf Introduction }
  \begin{enumerate}
    \item Background and motivation \\
      - personal background and lived experience, motivation
        (personal anecdote explaining the motivations behind why mathematics education needs reform, viewed and related in the perspective of a student)

      - background in postsecondary math education, including self-paced assessment ({\bf cite the few things we have})

      \begin{enumerate}
        \item find literature of how mathematics is typically taught ( citation needed, look Pedagogy of the oppressed by Freire [he calls this the banking model of education. explain this. then claim that banking model is being used in majority of education], Radical equations by Bob Moses)
        \item pedagogical developments (flipped classrooms (Mudd math), IBL (talk about Moore method. cite Brian Katz) -> all specific to math)
        \begin{enumerate}
          \item laying of the land of what nontraditional mathematics education looks like in postsecondary institutions
        \end{enumerate}
        \item gender inequity and underrepresentation / minoritization in math (AMS, CBMS, Sean Harper specifically about black male students, An invisible minority specifically about asian americans, Ursula Whitcher for women)
      \end{enumerate}

      - intro to self-paced assessment
  \end{enumerate}
  \item {\bf Mathematics is not fair }\\
  {\it rethink the chapter title, mathematics in the context is not fair, mathematics is not neutral. what implications does this title have?}
  \begin{enumerate}
    \item implicit bias by instructors
    \begin{enumerate}
      \item Harvard implicit bias study (ok if it's broad)\\
      -give disclaimer that this may or may not be related to mathematics
      \item John vs Jenny (implicit biases from minoritized is equivalent to those not)
    \end{enumerate}
    \item structural biases by institutions (colleges and universities)
    \begin{enumerate}
      \item access to resources, meals (textbooks, computers), private tutors, work and work study, trips home
      \item few faculty of color and women (role models and impact on students)
      \item few students of color and women (point to same resources as in introduction)
    \end{enumerate}
    \item reliance on SAT's, especially in mathematics (how that correlates to parents' incomes, relate that to how universities are reliant on this score)
    \item cultural obstructions to academic success*\\
    {\it mathematics culture specifically}
    \begin{enumerate}
      \item Alice and Bob
      \item cultural norms being imposed in the classroom (music, art, speech, dress); constructions by cis-het white male supremacy in mathematics {\bf Rochelle Gutierrez, Danny Martin, Nicole Joseph, Luis Leyva}
      \item Result -- underperformance
    \end{enumerate}
    \begin{itemize}
          \item Societal norms and impact of socio-economical differences in US\\
          - use literature to substantiate this, do not create your own arguments (show causation, not correlation)
          \item Whiteness in mathematics, western mathematics (Kumashiro, Bishop)
          \item Implicit and/or systemic biases in action
          \begin{enumerate}
            \item Microaggressions
            \item Model minority myth
          \end{enumerate}
          \item Outdated practices
          \begin{enumerate}
            \item Description of student demographics and historical changes/trends
            \item Correlation between student performances and student demographics
            \\ * need substantial evidence
          \end{enumerate}
    \end{itemize}
  \end{enumerate}
    \item{\bf Self-regulation in mathematics learning}
    \begin{enumerate}
      \item Different types of learning
      \begin{enumerate}
        \item Descriptions of mastery, inquiry (Moore method), flipped classrooms, group-based/active learning
        \item Focusing on how each impacted student learning
      \end{enumerate}
      \item Finding a commonality that can suggest a shift in the paradigm of learning
      \item Suggesting how self-regulation can reduce unfairness in mathematics learning
    \end{enumerate}
    \item {\bf A case study in self-regulation}
    \begin{enumerate}
      \item Introduction and description of self-paced assessments
      \item Qualitative description of setting and details of where and how the study took place
      \begin{enumerate}
        \item Honor code, establishment of trust
        \item Nature of course in study, number and demographics of students
        \item Structural differences
      \end{enumerate}
      \item Data Analysis
      \begin{enumerate}
        \item Tables and graphs of student backgrounds, performances
        \item need to talk to Prof Williams
        \item Qualitative student reports of differences in experience
      \end{enumerate}
      \item Discussion
      \begin{enumerate}
        \item Necessary adaptations and customizations that were
        \item Caveats and fallbacks of the case study
        \item What this suggests for us: a question to consider
      \end{enumerate}
    \end{enumerate}
\end{enumerate}
